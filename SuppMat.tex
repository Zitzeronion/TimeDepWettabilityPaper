\documentclass[amsmath,amssymb,showpacs,prl,superscriptaddress,notitlepage]{revtex4-1}

\usepackage{natbib}

\usepackage{graphics,graphicx,dcolumn,bm,fleqn,epic,eepic,float}
\usepackage{amssymb,amsmath,multirow,rotate,rotating,color}
\usepackage[utf8]{inputenc}

\usepackage[english]{babel}
\usepackage{caption}
\usepackage{subcaption}
\usepackage{tikz}
\usepackage{hyperref}
\hypersetup{
    colorlinks=true,
    linkcolor=blue,
    filecolor=magenta,      
    urlcolor=cyan,
}
%\usepackage[usenames,dvipsnames,svgnames,table]{xcolor}
\tikzset{fontscale/.style = {font=\relsize{#1}}}
\usetikzlibrary{calc}
\usepackage{color}
\usepackage{xcolor}
\usepackage{physics}
%\usepackage[colorlinks=true,linkcolor=blue]{hyperref}%
\hyphenation{title}

\begin{document}

\definecolor{pyblue}{HTML}{1F77B4}
\definecolor{pyorange}{HTML}{FF7F0C}
\definecolor{pygreen}{HTML}{2CA02C}
\definecolor{pyred}{HTML}{D62728}
\definecolor{jlblue}{rgb}{0.0,0.6056031611752245,0.9786801175696073}
\definecolor{jlorange}{rgb}{0.8888735002725198,0.43564919034818994,0.2781229361419438}
\newcommand*\Diff[1]{\mathop{}\!\mathrm{d^#1}}

\title{Controlling the dewetting morphologies of thin liquid films by switchable substrates: Supplemental Material}

\author{S. Zitz}
\email{zitz@ruc.dk}
 %Changed order: Work done while Stefan worked in Nuremberg
 \affiliation{Helmholtz Institute Erlangen-N\"urnberg for Renewable Energy,\\
  Forschungszentrum J\"ulich,
  F\"urther Strasse 248, 90429 N\"urnberg, Germany}%
  \affiliation{Department of Chemical and Biological Engineering, Friedrich-Alexander-Universit\"at Erlangen-N\"urnberg, F\"{u}rther Stra{\ss}e 248, 90429 N\"{u}rnberg, Germany}
  \affiliation{IMFUFA, Department of Science and Environment,\\ 
Roskilde University, Postbox 260, DK-4000 Roskilde, Denmark}%
\author{A. Scagliarini}%
\email{andrea.scagliarini@cnr.it}
 \affiliation{Institute for Applied Mathematics "M. Picone" (IAC), 
Consiglio Nazionale delle Ricerche (CNR),\\
Via dei Taurini 19, 00185 Rome, Italy}%
\affiliation{INFN, sezione Roma ``Tor Vergata'', via della Ricerca Scientifica 1, 00133 Rome, Italy}
\author{J. Harting}
\email{j.harting@fz-juelich.de}
 \affiliation{Helmholtz Institute Erlangen-N\"urnberg for Renewable Energy,\\
  Forschungszentrum J\"ulich,
  F\"urther Strasse 248, 90429 N\"urnberg, Germany}%
 \affiliation{Department of Chemical and Biological Engineering and Department of Physics, Friedrich-Alexander-Universit\"at Erlangen-N\"urnberg, F\"{u}rther Stra{\ss}e 248, 90429 N\"{u}rnberg, Germany}
\date{\today}

\maketitle
\section{Minkowski's structure metric $q_2$ in an extended parameter space}
\begin{figure}
    \centering
    \includegraphics[width=0.4\textwidth]{lam_256-128-64-51.pdf}
    \includegraphics[width=0.4\textwidth]{gamma15_deltaTheta.pdf}
    \caption{LEFT PANEL. Minkowski's structure metric $q_2$ for three different wavelengths $\lambda=64 h_0$, $\lambda=128 h_0$ and 
    $\lambda=256 h_0$ (as in figure 4 of the main text) and for $\Gamma=1.5$ (all other parameters are as in the main text). 
    The time interval during which $q_2 \approx 1$ signals the emergence of rivulets, also for $\lambda = 64 h_0$ which is comparable with the spinodal wavelength $\lambda_s \approx 70 h_0$. RIGHT PANEL. Comparison of the Minkowski's structure metric $q_2$ for $\delta\theta=5^{\circ}$ and $\delta\theta=10^{\circ}$, 
    $\Gamma = 15$ and $\lambda = 256 h_0$.}
    \label{fig:q2_difflambda}
\end{figure}
%\begin{figure}
%    \centering
%    \includegraphics[width=0.7\textwidth]{different_contrasts.pdf}
%    \caption{Comparison of the Minkowski's structure metric $q_2$ for different values of %$\delta\theta$, 
%    $\Gamma = 1.5$ and $\lambda = 256 h_0$.}
%    \label{fig:dif_contrast}
%\end{figure}
\noindent In this section we test the robustness of the observation of the rivulet 
state over a wider parameter space and in particular at changing: 1) the 
pattern wavelength $\lambda$ and 2) the heterogeneity amplitude $\delta \theta$.
To this aim we have run simulations with $\Gamma=1.5$, $\lambda/h_0 = 64, 128$, and with $\Gamma = 15$, $\lambda = 256 h_0$, $\delta \theta = 5^{\circ}$, respectively.
In Fig.~\ref{fig:q2_difflambda} (left panel) we report the measurements of the Minkowski's structure metric $q_2$, whose 
increase from zero up to a plateauing value of $q_2 \approx 1$ indicates the emergence of rivulets, for 
the different $\lambda$'s ($\lambda=256 h_0$, which is the value considered in the main text, is also reported 
for comparison).
Analogously, in the right panel, we report $q_2$ as a function of time for  
$\delta \theta = 5^{\circ}$  and $\delta \theta = 10^{\circ}$ (the value used in the main text). 
We see that the $q_2$ metric attains the value $q_2 \approx 1$, signalling the rivulet state, albeit over 
a time interval shorter than for $\delta \theta = 10^{\circ}$, i.e. the rivulets lifetime decreases with 
$\delta \theta$. This was somehow expected, since obviously (and also in the static case) 
the patterning looses effectiveness as the contact angle mismatch is reduced (see, e.g.~\cite{KonnurPRL2000}).

\newpage

\section{Droplet shape}

\noindent We investigate, here, how the local contact angle of droplets, formed on the more
hydrophilic patches after dewetting, depends on the pattern wavelength. We recall, in fact,
that the patterning is such that the contact angle is not piecewise constant, but varies
with continuity.
The droplet shape is determined by minimization of the total interfacial energy:
\begin{equation}\label{eq:energy}
E = \gamma_{\text{lg}} A_{\text{lg}} + \int_{A_{\text{sl}}} (\gamma_{\text{sl}} - \gamma_{\text{sg}})d \sigma,
\end{equation}
where $A_{\text{lg}}$ and $A_{\text{sl}}$ are the liquid/gas and solid/liquid interface areas, and $\gamma_{\text{lg}}$, $\gamma_{\text{sl}}$,
$\gamma_{\text{sg}}$ are the liquid/gas, solid/liquid and solid/gas interface energies 
per unit area \cite{WuSM2020}. 
In particular, $\gamma_{\text{lg}} \equiv \gamma$ is the surface tension.
Setting $A_{\text{lg}} \equiv A$ and $A_{\text{sl}} \equiv S$, by Young's equation 
$\cos \theta = \frac{\gamma_{\text{sg}} - \gamma_{\text{sl}}}{\gamma}$,
Eq.~(\ref{eq:energy}) can be rewritten as:
\begin{equation}\label{eq:energyoung}
\tilde{E} \equiv \frac{E}{\gamma} = A - \int_S \cos \theta \text{d}x \text{d}y.
\end{equation}
Droplets will form around minima of the contact angle pattern
\begin{equation}\label{eq:contact}
\theta(x,y) = \theta_0 + \delta \theta \sin(q_{\theta} x) \sin (q_{\theta} y) \qquad q_{\theta} = \frac{2\pi}{\lambda},
\end{equation}
namely $(x_n,y_n) = \left((2n+1)\frac{\lambda}{4},(2n+3)\frac{\lambda}{4}\right)$, with
$n=0,\pm 1, \pm 2,\dots$.
If we consider large wavelengths ($q_{\theta} h_0 \ll 1$) and heterogeneity
($\delta \theta \ll 1$) such that the contact angle gradients are small, the
expression (\ref{eq:contact}) can be expanded as
\begin{equation}\label{eq:contact2}
\theta(x,y) \approx \theta_m + \delta \theta q_{\theta}^2 ((x-x_n)^2 + (y - y_n)^2) + o(|\mathbf{x}-\mathbf{x}_n|^2),
\end{equation}
where $\theta_m = \theta_0 - \delta \theta$.
This local radial symmetry allows to approximate the equilibrium droplet shape as a spherical cap of
height $h$ and base radius $a$, whose area is $A = \pi(a^2 + h^2)$; inserting the expression for
$A$ and (\ref{eq:contact2}), neglecting higher than second order terms, in (\ref{eq:energyoung}) gives:
\begin{equation}\label{eq:energy3}
  \tilde{E}(h,a) = \pi(a^2 + h^2) -
  \int_{\mathcal{C}_a(\mathbf{x}_n)} \cos\left[\theta_m + \delta \theta q_{\theta}^2 ((x-x_n)^2 + (y - y_n)^2)\right]
    \text{d}x\text{d}y,
\end{equation}  
where
$\mathcal{C}_a(\mathbf{x}_n)=\{(x,y) \in [0, L]^2|(x-x_n)^2 + (y-y_n)^2 \leq a^2\}$
is the circle of centre $\mathbf{x}_n$ and radius $a$. 
Due to global volume conservation and assuming the droplets to be monodisperse, the droplet
volume is $V_d = h_0L^2/N_d = h_0 \lambda^2/2$,
where $N_d$ is the number of droplets, which equals the number of
minima of (\ref{eq:contact}) in the domain $[0, L]^2$, i.e. $N_d = 2(L/\lambda)^2$.
Enforcing the volume of the spherical cap to be equal to $V_d$ relates $h$ and $a$ by
\begin{equation}
\frac{\pi h}{6}(3a^2 + h^2) \approx \frac{\pi}{2} a^2 h =  \frac{h_0 \lambda^2}{2},    
\end{equation}
in the ``lubrication approximation'' $h \ll a$, whence
\begin{equation}
  h \approx \left(\frac{h_0}{\pi}\right) \left(\frac{\lambda}{a}\right)^2.
\end{equation}  
Inserting the latter expression in (\ref{eq:energy3}) and performing the integral, the
energy (that we indicate now as $E(a)$ to lighten the notation) reads:
\begin{equation}\label{eq:energy4}
  E(a) = \pi \left(a^2 + \frac{h_0^2 \lambda^4}{\pi^2 a^4} \right) - \frac{2\pi}{\delta \theta q_{\theta}^2}
  \left[\sin\left(\theta_m +\frac{\delta \theta}{2}q_{\theta}^2a^2\right) - \sin \theta_m \right].
\end{equation}  
We expand, then, the sine in the second term up to second order in $\delta \theta$ (such that the
energy is first order) and we finally get:
\begin{equation}\label{eq:energyfin}
  E(a) \approx  \pi \left((1-\cos \theta_m) a^2 + \frac{h_0^2 \lambda^4}{\pi^2 a^4} \right)
  + \delta \theta \frac{\pi}{4}\sin \theta_m q_{\theta}^2 a^4 \equiv E_0(a) + \delta \theta E_1(a).
\end{equation}  
The minimum condition $\frac{\partial E}{\partial a} = 0$ \cite{footnote} yields
\begin{equation}\label{eq:minim}
2\pi (1-\cos \theta_m)a^6 - 4h_0 \lambda^4 + \delta \theta \pi^2 \sin \theta_m q_{\theta}^2 a^8 =0.
\end{equation}  
The solution of (\ref{eq:minim}) at zero-th order in $\delta \theta$ is
\begin{equation}\label{eq:a0}
  a_0 = \left[\frac{2h_0\lambda^4}{\pi^2(1-\cos \theta_m)}\right]^{1/6};
\end{equation}
to this order the droplet contact angle $\tan (\theta_d/2) = h/a$ reads
\begin{equation}
\tan \left(\frac{\theta_d}{2}\right) = \frac{h}{a} \approx
\left(\frac{h_0}{\pi}\right)\frac{\lambda^2}{a_0^3} =
\left(\frac{1-\cos \theta_m}{2}\right)^{1/2}\frac{\lambda^2}{(\lambda^{2/3})^3} \approx \frac{\theta_m}{2} \quad \Rightarrow \quad \theta_d \approx \theta_m,
\end{equation}
i.e., it does not depend on $\lambda$. We move, then, to the next order. We seek a solution to
(\ref{eq:minim}) in the form $a = a^{(0)} + \delta \theta a^{(1)} + \dots$, where $a^{(0)} \equiv a_0$;
at the first order in $\delta \theta$ we get:
\begin{equation}
a^{(1)} = -q_{\theta}^2 a_0^3\frac{\sin \theta_m}{6(1-\cos \theta_m)},
\end{equation}
such that the correction to the droplet contact angle provides:
\begin{equation}
  \tan \left(\frac{\theta_d}{2}\right) \approx \frac{h_0 \lambda^2}{\pi a_0^3}
  \left(1+\delta \theta q_{\theta}^2 a_0^2 \frac{\sin \theta_m}{12(1-\cos \theta_m)}\right)
  \quad \Rightarrow \quad \theta_d \approx
  \theta_m \left(1+\frac{\delta \theta}{\theta_m}\frac{b}{\lambda^{2/3}}\right),
\end{equation}  
where $b = \left(\frac{32 h_0^2 \pi^4}{27 \theta_m^5}\right)^{1/3}$,
i.e. the droplet contact angle grows at decreasing pattern wavelength as $\theta_d \sim \lambda^{-2/3}$.

\begin{thebibliography}{99}

\bibitem{KonnurPRL2000} R. Konnur, K. Kargupta and A. Sharma, {\it Phys. Rev. Lett.} {\bf 84}, 931--934 (2000).
\bibitem{WuSM2020} Y. Wu, F. Wang, S. Ma, M. Selzer and B. Nestler,
{\it Soft Matter} {\bf 16}, 6115--6127 (2020).
\bibitem{footnote} It can be easily checked that $\frac{\partial^2 E}{\partial a^2} > 0$ for $a>0$.
  
\end{thebibliography}

\end{document}
