\documentclass[12pt,english]{article}

\usepackage{natbib}

\usepackage{graphics,graphicx,dcolumn,bm,fleqn,epic,eepic,float}
\usepackage{amssymb,amsmath,multirow,rotate,rotating,color}
\usepackage[utf8]{inputenc}

\usepackage[english]{babel}
\usepackage{caption}
\usepackage{subcaption}
\usepackage{tikz}
\usepackage{hyperref}
\hypersetup{
    colorlinks=true,
    linkcolor=blue,
    filecolor=magenta,      
    urlcolor=cyan,
}
%\usepackage[usenames,dvipsnames,svgnames,table]{xcolor}
\tikzset{fontscale/.style = {font=\relsize{#1}}}
\usetikzlibrary{calc}
\makeatother

\newcommand{\figref}[1]{Fig.~\ref{fig:#1}}
\newcommand{\eqnref}[1]{Eq.~(\ref{eq:#1})}
\newcommand{\ts}{\textsuperscript}

\definecolor{tuered}{RGB}{214,0,74}

\newcommand{\todo}[1]{\textbf{\textcolor{tuered}{ TODO: #1}}}

\newcommand{\vectornorm}[1]{\left|\left|#1\right|\right|}

\renewcommand{\vec}[1]{\mathbf{#1}}
\newcommand{\uvec}[1]{\hat{\vec{#1}}}
\newcommand{\tensor}[1]{\mathbf{#1}}

\definecolor{pyblue}{HTML}{1F77B4}
\definecolor{pyorange}{HTML}{FF7F0C}
\definecolor{pygreen}{HTML}{2CA02C}
\definecolor{pyred}{HTML}{D62728}

\newcommand{\JH}[1]{\textcolor{blue}{JH: #1}}
\begin{document}

\vspace*{-3cm}

\noindent Dear Editor,\\

we thank you for sending us the constructive reports on our manuscript. We are glad to see 
that the Referees agree that the work is interesting and scientifically relevant, with one of them 
(Referee B) inclined to accept it.

In particular, we would like to underline that not a single 
concern was raised on the physics, namely on the reliability of the results.
However, we acknowledge that certain points were not explained and/or discussed adequately, and therefore we invested a substantial effort in clarifying all
doubts and concerns.
Our major revision takes into account all Referees' reasonable and constructive remarks, including novel 
results from new simulations and theoretical derivations (see the new Supplementary Material).

Let us, please, stress further how and why we deem that the present paper meets the Physical
Review Letters criteria of impact and innovation (interest, as we already noted, was acknowledged by the Referees).
We found a novel effect in the dewetting on substrates with properly tailored time dependent properties and 
suggest how to exploit it to control the dewetting morphologies. Motivated by a Referee's suggestion, we even propose a possible experimental realisation.
We think that the impact of this discovery is out of question, as witnessed also by the various recent publications on dewetting and closely related problems, which appeared in PRL 
(just to cite but a few recent ones: J.C. Fern\'andez-Toledano et al, ``How wettability controls nanoprinting'',
{\it Phys. Rev. Lett.} {\bf 124} 224503 (2020); A.A. Pahlavan et al,
``Evaporation of binary-mixture liquid droplets: the formation of picoliter pancakelike shapes'', {\it Phys. Rev. Lett.} {\bf 127}, 024501 (2021); 
C. Clavaud et al, ``Modification of the fluctuation dynamics of ultrathin wetting films'', {\it Phys. Rev. Lett.} {\bf 126}, 228004 (2021)).

We would like to point out a few flaws in the referee reports, which in our opinion have led to a misjudgement of our work. Two referees base their criticism on a clear misconception of the aim and background of the paper:
First, Referee A ignores the concept of disjoining pressure, confuses dewetting with prewetting and concepts from the physics of free surface flows to wetting.
Second, the main ``argument'' of Referee C against publication is that he/she does not see ``how would the transport of a fluid by rivulets 
surpass the transport by droplets''. However, as it is clearly stated in the title of our manuscript, the paper is about controlling morphologies, not about transport. 

We believe that our manuscript, in its new version, deserves to be brought again to your attention as a contribution to Physical Review Letters and we are 
confident that it might now be suitable for publication.

~\\
With our best regards,\\
Stefan Zitz, Andrea Scagliarini, and Jens Harting

\end{document}
