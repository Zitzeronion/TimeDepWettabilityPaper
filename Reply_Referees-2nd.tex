\documentclass[12pt,english]{article}

\usepackage{natbib}

\usepackage{graphics,graphicx,dcolumn,bm,fleqn,epic,eepic,float}
\usepackage{amssymb,amsmath,multirow,rotate,rotating,color}
\usepackage[utf8]{inputenc}

\usepackage[english]{babel}
\usepackage{caption}
\usepackage{subcaption}
\usepackage{tikz}
\usepackage{hyperref}
\hypersetup{
    colorlinks=true,
    linkcolor=blue,
    filecolor=magenta,      
    urlcolor=cyan,
}
%\usepackage[usenames,dvipsnames,svgnames,table]{xcolor}
\tikzset{fontscale/.style = {font=\relsize{#1}}}
\usetikzlibrary{calc}
\makeatother

\newcommand{\figref}[1]{Fig.~\ref{fig:#1}}
\newcommand{\eqnref}[1]{Eq.~(\ref{eq:#1})}
\newcommand{\ts}{\textsuperscript}

\definecolor{tuered}{RGB}{214,0,74}

\newcommand{\todo}[1]{\textbf{\textcolor{tuered}{ TODO: #1}}}

\newcommand{\vectornorm}[1]{\left|\left|#1\right|\right|}

\renewcommand{\vec}[1]{\mathbf{#1}}
\newcommand{\uvec}[1]{\hat{\vec{#1}}}
\newcommand{\tensor}[1]{\mathbf{#1}}

\definecolor{pyblue}{HTML}{1F77B4}
\definecolor{pyorange}{HTML}{FF7F0C}
\definecolor{pygreen}{HTML}{2CA02C}
\definecolor{pyred}{HTML}{D62728}

\newcommand{\JH}[1]{\textcolor{blue}{JH: #1}}
\begin{document}

% The referee PDF has been generated from r239 -- 'make referee' command.

\section*{Reply to 2nd report of Referee A}
\begin{itemize}  
\item[ \textbf{\underline{Comment 1.}}]
{ 
With the response given to me by the authors, I think that the
research is probably valid. I still fail to see why I should be very
excited about the findings, so I think that the manuscript would be
better suited for a more specialized journal, like Physical Review E.
The reason for this recommendation is that I can see what the
patterned substrate does, in the sense that the wavelength competes
with that of dewetting. That droplets can mode (move?) in a wettability
gradient has also been clear for a long time. I am wondering what is
there to be gained by combining these ideas, and I find the rationale
of the manuscript to be not extremely compelling. In my view, this
would be needed for PRL. I am sorry that I cannot be more positive.
}

\item[ \textbf{{Answer}}]
{
We thank the Referee for taking his/her time to read our 
manuscript once more and his/her subjective opinion. It is good to read 
that he/she thinks that our research is valid and that he/she finds the 
manuscript compelling to some extent. 

We would like to stress once more that 
the main focus of the paper is not dewetting or droplet movement in a {\it stationary} wettability gradient, but dewetting 
morphologies over a {\it switchable substrate}, i.e. whose wettability pattern 
is {\it time-dependent}, whence the title 'Controlling dewetting morphologies of 
thin liquid films by switchable substrates'.

In this context, we have discovered a new morphological transition which we have fully characterized 
and theoretically interpreted in terms of the control parameter $\Gamma$.
This said, even in the static pattern case, although the dewetting problem was previously studied, we 
reported for the first time, together with a theoretical explanation, the absolutely non-trivial dependence of the rupture times on the pattern wavelength.
Let us remark again that a precise control of the liquid-vapour interface during a dynamic process is a highly desired feature for many applications. We did show that the liquid-vapour 
interface even after dewetting can be increased significantly due to
the discussed morphological transition. 

More in general wetting, dewetting and the dynamics of droplets on complex substrates 
are state-of-the-art topics of interest to a broad audience, both in experimental and 
theoretical fluid physics. 
We also believe that a deeper understanding of the dynamics of thin liquid films on switchable 
substrates, which this paper contributes to, is essential to push forward the current reach of microfluidic technology.
}
\end{itemize}  

\section*{Reply to 2nd report of Referee B}
\begin{itemize}  
\item[ \textbf{\underline{Comment 1.}}]
{ 
The authors have responded convincingly to my concerns and addressed
them sufficiently in the revised manuscript. I have also read the
responses to the other reports and I feel that the authors have
addressed these recommendations adequately. In view of this, and my
original favorable assessment, I recommend this manuscript for
publication.

I have just two remarks that are related to concern 5 in my original
report:

1. Most importantly, the authors have added footnote 26, which I find
to be unnecessary, but notice that the curvature has the denominator
to the power of 3/2.

2. The film thickness-dependent surface tension referred to in my
report is not related to the trivial small gradient approximation, and
appears whether the slope is small or large. It is a second order
effect which decays according to d Pi(h) / dh, and could be relevant
for thin films and helpful to address the contact line problem, but
not likely to change the results qualitatively.
}

\item[ \textbf{{Answer}}]
{
We thank the Referee for clarifying his/her question and we apologize for 
our misunderstanding (we have removed the footnote and the comment on 
the small gradient expansion of the curvature). 
We agree that, when thermal fluctuations are relevant,
it is important to consider the renormalized (i.e., height dependent) surface tension,
especially when it comes to the initial growth of the instability. It is, on the other 
hand, reasonable to expect, as the Referee remarks, that in terms of the dewetting 
morphology (and its relation to the time-dependence of the substrate) the phenomenology
should not be particularly affected, at least qualitatively.
}
\end{itemize}  


\section*{Reply to Referee D}
We thank the new Referee for his/her report on our manuscript. 
We are glad to read that he/she acknowledges the systematic approach we have taken and are happy to see that he/she finds our results interesting. 
We understand the concerns raised by the referee and wonder why the recommendation is negative.
Hereafter, we provide answers to the various questions/comments raised. 
All changes in the revised version are reported in red color.

\begin{itemize}

\item[ \textbf{\underline{Comment 1.}}]
{ 
It seems impossible to construct the patterns described by Equation (3) in a real experiment, making the current research less attractive.
}

\item[ \textbf{{Answer}}]
{
We are very surprised by this comment and do not agree with the assessment of the referee. 
% For static patterns it is well known how to pattern substrates in a precise and controlled way using, e.g. lithography. 
It is true that the novelty of the here presented work is strongly coupled to both the temporal and spatial component of the wettability and as such to equation (3).
We are certain that switchable substrates offer this kind of control, and for photoswitchable materials spiropyran should be a perfect candidate~\cite{keyvanradSpiropyranbasedAdvancedPhotoswitchable2022}. 

Not only do we present our numerical and analytical results in this letter, but we also supply a potential realization for a lab experiment. 
We provide a schematic setup using a photoswitchable substrate and a digital multimirror device. 
As we are well aware that this is not sufficient for an experiment we further identify the relevant length and time scales for both the film and the apparatus.
% Hoping that experimentalist confirm transition we supply a way to construct pattern Eq. (3).
%We therefore have a hard time understanding why it seems impossible to the referee to construct the pattern equation %(3), as he/she does not give us a single increment of information in his/her comment.
}

\item[ \textbf{\underline{Comment 2.}}] 
{
This work is completed based on the lubrication equation for the thin film flows.
From theoretical/mathematical aspects, the novelty of this work, in my opinion, is not strong enough. 
This lubrication model employed here was proposed about 40 years ago and has been thoroughly investigated~\cite{RevModPhys.69.931, RevModPhys.81.1131}. 
There are no improvements or modifications to the model in this work. 
Additionally, the model for the rupture time comes from a simple scaling analysis, which is not very convincing when considering other classical mathematical approaches, such as linear instability analysis, similarity transformation, etc.
}

\item[ \textbf{{Answer}}]
{
We find it puzzling that the referee arrives at the conclusion that working on a theory which is 40 years old does not allow for the novelty attributed to PRL.
A good counter example is: M. Wilczek et al,
"Sliding drops – ensemble statistics from single drop bifurcations", {\it Phys. Rev. Lett.} {\bf 119}, 204501 (2017), in which Wilczek et al. use the thin film equation on an inclined plane.
Similar arguments can be constructed about the Navier-Stokes equation which is 180 years old, still there is a lot of research dedicated towards, e.g., turbulence.

We agree with the referee that from a purely mathematical point of view lubrication theory is untouched.
There is no reason or advantage to add artificial complexity to a well working theory which has been developed for this singular purpose.
However, when it comes to dynamical effects and non-linear behaviour of thin films, the topic this letter revolves around, mathematics falls short and only numerical simulation or experiments offer deeper insights, which we also outline in the first paragraph of the introduction:

``\textit{While a consistent body of theoretical/computational work was devoted to processes on static heterogeneous substrates, the time dependent case is still almost unexplored, with few exceptions focusing on single droplet spreading and sliding~\cite{GrawitterStark1,GrawitterStark2,ThieleHartmann} or limited to analysing the linear regime~\cite{suman2006dynamics}.}''

To this end we use a novel lattice Boltzmann method to solve the thin film equation at time scales where initial small perturbations have grown to similar sizes as $h_0$.
It is well known that the rupture time marks the end of the linear regime and linear instability theory as suggested by the referee, which has been used in our previous work~\cite{PhysRevE.104.034801}. 
However, it is simply not able to explain our observations, as $\tau_r$ is dependent on $\Gamma$. 
Observing the morphological transition during dewetting from a droplet to a rivulet state is by no means covered by linear analysis. 
It is a dynamic competition of relevant velocities which to the best knowledge of the authors has not been discussed or shown before.
}

\item[ \textbf{\underline{Comment 3.}}]
{
The mobility function in Equation (1) contains the slip length $\delta$. 
But there is no further information in this work. 
Is the slip length of a patterned substrate constant? 
If so, it is quite unreasonable because a hydrophobic substrate is usually more slippery than a hydrophilic one. 
Moreover, previous works had showed that the slip is crucial to the interface dynamics~\cite{zhang_sprittles_lockerby_2021, zhao_zhang_si_2023}, which should be taken into account.
}

\item[ \textbf{{Answer}}]
{
We thank the Referee for this remark. It is of course true that the slip length 
should grow with the contact angle, on the other hand it is also well known that for low contact angle (actually up to $\sim 100^{\circ}$) such dependence is relatively weak, 
as it is expected to follow the relation $\lambda \propto (1+\cos \theta)^{-2}$
(see Huang et al, {\it Phys. Rev. Lett.} {\bf 101}, 226101 (2008)). The ratio 
of the slip lengths on patches of contanct angles $10^{\circ}$ and $30^{\circ}$
(which is the case of our simulations), for instance, would then be $\approx 0.9$,
such that the slip length should have been varied between $0.95$ and $1.05$ lbu. We deemed, then, reasonable to consider it approximately constant. 
To provide an idea, in reference~\cite{zhang_sprittles_lockerby_2021}, for instance, effects were appreciated 
when slip lengths were two to five times larger than each other. 
}

\item[ \textbf{\underline{Comment 4.}}]
{ 
According to the fluid parameters displayed in the supplemental material, the initial thickness of the film is about 4 nm. 
Thermal fluctuations had been shown to play a key role at this scale~\cite{PhysRevLett.99.114503, PhysRevLett.126.228004}, which might affect the dynamics seriously.
}

\item[ \textbf{{Answer}}]
{
We do not say that the initial film thickness is $4 \, \text{nm}$, we just considered the experiments of Becker et al (where the initial film thickness is $\sim 4 \, \text{nm}$) as a reference, since they provide experimental values of quantities (such as the Hamaker constant) that enable the comparison 
with the numerics. The effect of thermal fluctuations, though, is known to essentially show up as 
a speed up of the growth of the instability, but it does not affect particularly the dewetting morphology (as shown in Becker et al, {\it Nat. Mater.} {\bf 2}, 59--63, who made direct comparison 
of the numerical simulation of a lubrication model with experiments).
}


\end{itemize}


\bibliographystyle{abbrv}
\bibliography{Ref}

\end{document}
