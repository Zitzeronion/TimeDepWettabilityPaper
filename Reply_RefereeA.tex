\documentclass[12pt,english]{article}

\usepackage{natbib}

\usepackage{graphics,graphicx,dcolumn,bm,fleqn,epic,eepic,float}
\usepackage{amssymb,amsmath,multirow,rotate,rotating,color}
\usepackage[utf8]{inputenc}

\usepackage[english]{babel}
\usepackage{caption}
\usepackage{subcaption}
\usepackage{tikz}
\usepackage{hyperref}
\hypersetup{
    colorlinks=true,
    linkcolor=blue,
    filecolor=magenta,      
    urlcolor=cyan,
}
%\usepackage[usenames,dvipsnames,svgnames,table]{xcolor}
\tikzset{fontscale/.style = {font=\relsize{#1}}}
\usetikzlibrary{calc}
\makeatother

\newcommand{\figref}[1]{Fig.~\ref{fig:#1}}
\newcommand{\eqnref}[1]{Eq.~(\ref{eq:#1})}
\newcommand{\ts}{\textsuperscript}

\definecolor{tuered}{RGB}{214,0,74}

\newcommand{\todo}[1]{\textbf{\textcolor{tuered}{ TODO: #1}}}

\newcommand{\vectornorm}[1]{\left|\left|#1\right|\right|}

\renewcommand{\vec}[1]{\mathbf{#1}}
\newcommand{\uvec}[1]{\hat{\vec{#1}}}
\newcommand{\tensor}[1]{\mathbf{#1}}

\definecolor{pyblue}{HTML}{1F77B4}
\definecolor{pyorange}{HTML}{FF7F0C}
\definecolor{pygreen}{HTML}{2CA02C}
\definecolor{pyred}{HTML}{D62728}

\newcommand{\JH}[1]{\textcolor{blue}{JH: #1}}
\begin{document}

% The referee PDF has been generated from r239 -- 'make referee' command.

\section*{Reply to Referee A}
We thank the Referee for his/her report on our manuscript. 
We are glad to read that he/she finds it of potential interest and in principle 
suitable of publication, if a thorough revision is performed. 
We substantially improved and revised the paper for the resubmission and hope that we were able to clarify the obscure points, following all the Referee's remarks.
Hereafter we provide answers to the various questions/comments raised. All changes in the revised version are reported in red colour.

\begin{itemize}

\item[ \textbf{\underline{Comment 1.}}]
{ 
I am unable to follow most of the arguments made with respect to the disjoining pressure. 
For me, one needs to integrate the disjoining pressure to arrive at an effective interface potential, and 
I do not really know how to go from the effective interface potential to the surface tension. 
The equation (3) therefore merits much more discussion.
}

\item[ \textbf{{Answer}}]
{
We did not give, in fact, any particular argument with respect to the disjoining pressure, since 
we assumed that it is a very well established concept employed in any theoretical and computational study of thin liquid film hydrodynamics in the lubrication approximation framework.
We also assumed that an equation like Eq.(3) (now Eq.(2)), which is approximately twenty-five years old \cite{SCHWARTZ1998173,Mitlin94}, did not deserve a lengthy discussion.
However, our assumptions were probably erroneous and, therefore, we give, here, some more explanation. 
Let us, first, make a step back and underline a fundamental difference between the physics of {\it one free interface} and that of a {\it film}:
the latter is a layer of liquid confined between {\it two} interfaces. When the two interfaces are brought close enough to each other (the layer is thin), they interact.
The disjoining pressure stems from this interaction. Thermodynamically, it represents minus the derivative of the free energy difference, per unit surface, between 
the film and the bulk of the same liquid, or, equivalently, the pressure difference inside and outside the film 
\cite{Deryaguin1940,DeryaguinChuraev1978}.
In the case of wetting, there can be three types of interfaces: liquid/air, liquid/solid (the substrate) and, 
if the substrate is only partially wetted, solid/air, thus giving rise to three-phase contact lines. 
The usually called {\it surface tension} is the energy cost of forming a liquid/air interface of unit area and its only knowledge is, therefore, not enough in our case. 
The Young's condition \cite{Young1805,DeGennes1985} connects the surface tension with the analogous energies per unit
area associated with the solid/liquid ($\gamma_{\text{sl}}$) and solid/gas ($\gamma_{\text{sg}}$) interface as
\begin{equation}
\cos \theta = \frac{\gamma_{\text{sg}}-\gamma_{\text{sl}}}{\gamma},
\end{equation}
where $\theta$ is the contact angle.
So, there is no need (and no way) to go from the disjoining pressure (or the interface potential) to the surface tension, since the two quantities carry complementary physical information,
and both are required to determine the dynamics of dewetting.
Of course, on the other hand, the disjoining pressure depends also on the surface tension, which appears, in fact, in its expression (3). 
We think, though, that part of the confusion might be due to the fact that we do not define the symbol of the surface tension at its first occurrence, but only later. 
We have now fixed this and we apologize for it.
}

\item[ \textbf{\underline{Comment 2.}}] 
{
The same holds for the time scale. 
The viscous time scale is well defined, but necessitates a length scale in addition to the 
capillary velocity. It is unclear why the authors choose the length scale that they use in the paper.
Subsequently, they replace the surface tension by a complicated partial derivative of the 
disjoining pressure; here the same remark as above holds.
}

\item[ \textbf{{Answer}}]
{
We apologize if we have not been thorough in the discussion of the relevant scales, which indeed deserves some clarification. 
Thin film hydrodynamics is an intrinsically multiscale phenomenon, whereby there are at least two characteristic length scales: the mean film thickness (let us call it $h_0$)
and an horizontal length scale $\ell$ (which in our case can be, for example, the pattern wavelength, $\lambda$), where, typically $h_0 \ll \ell$. We chose to use the mean 
film thickness as a scale, because it is a relevant length that is common to all the simulations.
In the same spirit, we chose as a reference time $t_0(\theta)$, which is not exactly the capillary time (though being related to it). 
$t_0$, as we explain in the text, is, in fact, the characteristic time of growth of the most unstable mode in the spinodal dewetting on a homogeneous
substrate of contact angle $\theta$. In spite of the substrate being patterned in our case, it is possible to define an effective $t_0$ in terms 
of, e.g., the mean contact angle on the pattern, i.e. $t_0(\theta_0)$. 
We deem the so defined characteristic time as an appropriate scale,
since it does not depend on the pattern wavelength and speed. Actually, one can see that it captures well the time scales of the dewetting process
from figure 2, where it is shown that the rupture times are of the order of fraction to units of $t_0$.
We recognize, on the other hand that in this perspective the choice of $v_0 = \lambda/t_0$ as a reference velocity looks a bit awkward.
A more natural candidate would be, indeed, $v_0=\lambda_s/t_0$, where $\lambda_s$ is the spinodal 
characteristic length corresponding to $\theta_0$. We have therefore decided to use the latter in the revised version of the manuscript.
We have also rephrased the corresponding paragraph in the {\it Method} section as follows: \\

\textcolor{red}{Lengths and time scales will be expressed, respectively, in units of the mean film height, $h_0$ (which is constant in time, due to mass conservation), and of $t_0 = \frac{3\mu}{\gamma h_0^3 q_0^4}$, the inverse growth rate of the most unstable mode, whose wavenumber is $q_0$, of a spinodally dewetting film. On a uniform substrate, with constant contact angle $\theta^{(u)}$, the wavenumber reads 
$(q^{(u)}_0)^2 = h_{\ast}^{-2}(1-\cos \theta^{(u)})f^{\prime}(h_0/h_{\ast})$ \cite{Mecke_2005,PhysRevE.100.023108}. In our patterned case, we define 
$q_0^2=h_{\ast}^{-2}(1-\cos\theta_0)f^{\prime}(h_0/h_{\ast})$.
Correspondingly, we choose as a velocity scale $v_0 = \lambda_s/t_0$, where $\lambda_s = 2\pi/q_0$.}

}

\item[ \textbf{\underline{Comment 3.}}]
{ 
Incidentally, for a 'normal' fluid such as water, the capillary velocity is $\sim 70\mbox{m}/\mbox{s}$, 
so that very, very rapidly the fluid inertia becomes important. 
This is indeed what one usually observes for the formation or coalescence of drops, so that at the very least this should be discussed and the Reynolds number calculated.
}

\item[ \textbf{{Answer}}]
{
Such a high value corresponds to the ratio of (surface tension)/(dynamic viscosity) for water, i.e. $v = \gamma/\eta \approx 70 \mbox{m}/\mbox{s}$.
This expression, though, is only representative of the capillary velocity, again, for {\it free surface flows}, as, e.g., in the coalescence of 
two droplets in the bulk of another fluid. 
In the case wetting/dewetting, 
the phenomenology is complicated by the liquid/substrate (and vapour/substrate) interaction. 
In such a situation the relevant velocity scale is set by the characteristic speed of contact line 
motion which can be estimated from a viscocapillary balance as 
$\sim \varepsilon^3 \gamma/\eta$, where $\varepsilon = h_0/L \ll 1$ is the film thickness parameter 
\cite{RevModPhys.69.931,doi:10.1146/annurev-fluid-011212-140734}. 
The typical velocities in dewetting problems are, then, much smaller than the ratio $\gamma/\eta$.
Moreover, in many actual applications, as well as in the majority of experiments, the 
working fluids are in general much more viscous than water (e.g., polymeric liquids), thus making
the characteristic velocity even smaller.
For the dewetting of a millimetric film of micrometric thickness reatracting into a single droplet, for instance, experiments have measured that 
the typical velocity is $\sim 1 \mbox{mm}/\mbox{s}$, which gives a Reynolds number $Re \approx 10^{-3}$ \cite{Edwards2016}. 
In our problem, there are two relevant characteristic velocities: the retraction speed, 
$U_{\Theta}$, and 
the pattern speed, $v_{\theta}$. When their ratio $\Gamma = v_{\theta}/U_{\Theta} < 1$, the retraction speed dominates and fixes the Reynolds number which can be, then, evaluated as 
$Re = U_{\Theta}h_0/\nu$ \cite{RevModPhys.69.931}, whereas for $\Gamma > 1$, $Re = v_{\theta} h_0/\nu$. 
In any case, the Reynolds number stays always below $\approx 0.2$.
We have added the following comment at the end of the {\it Method} section:\\

\textcolor{red}{The Reynolds number, $Re$, is of the order of $Re \sim 10^{-2}$ on the 
static substrate and never exceeds the value $\approx 0.2$ in the time dependent case.}

}

\item[ \textbf{\underline{Comment 4.}}]
{ 
In addition, there is a lot of old literature that relates the dewetting to the disjoining pressure, and shows that the nucleation of a hole in a wetting film is much more difficult than the nucleation of such a wetting film; there are experiments on this by Rutledge and Taborek, theory by Schick and Taborek and Bausch, Blossey and Indekeu.
}

\item[ \textbf{{Answer}}]
{
We are well aware of the literature quoted by the Referee, but these are works 
about {\it prewetting}, namely the phenomenon of {\it nucleation} (condensation)
of films/droplets on solid surfaces confining a critical liquid-vapour mixture
{\it as a function of temperature}. So, we are a bit puzzled about what such literature may have to do with the {\it dewetting} of a film on a chemically patterned substrate. Here, in fact, we are modelling a thin liquid film which is 
thermodaynamically stable against evaporation (i.e. one can consider that 
we are at a - fixed - temperature below the critical one).
}

\item[ \textbf{\underline{Comment 5.}}]
{ 
The explanations also lack a discussion on time, length, and energy scales. 
I have no clue what the values mean ``the numerical values, in LB units, are set to $\gamma= 0.01$ and $\mu = 1/6$''?. 
}

\item[ \textbf{{Answer}}]
{
Length and time scales, as discussed in Comment 2 and as mentioned also in the original version of the manuscript, 
are made dimensionless with the mean film height and the characteristic time $t_0$. 
These two quantities 
have a clear physical meaning and can, therefore, be used to compare with 
experimental setups. In particular, the expression of $t_0$, through its dependence 
on $q_0$ (the wavenumber of the most unstable mode) contains the details 
of the disjoining pressure, such as the Hamaker constant $A_H$. For instance, let us consider the problem studied in \cite{becker2003complex}, the dewetting of a polystyrene film of thickness $h_0 \approx 4 \, \text{nm}$ deposited on a silicon 
dioxide substrate. The Hamaker constant for this system is 
$A_H  = 2.2 \times 10^{-20} \, \text{J}$ and the surface tension is 
$\gamma = 0.03 \, \text{N}/\text{m}$, such that, 
from equation (3) of \cite{becker2003complex} and 
from the definition of $q_0^2 = \frac{1}{2\gamma} \Pi^{\prime}(h_0)$, 
we get (neglecting the short range part of the interface potential)
$$
q_0^2 \approx \frac{A_H}{4\pi \gamma h_0^4} \approx 2.5 \times 10^{-4} \, \text{nm}^{-2}.
$$
The characteristic time of the growth of the instability, then, reads \cite{PhysRevLett.99.114503} (the dynamic viscosity of polystyrene is 
$\mu = 1.2 \times 10^4 \, \text{Pa} \cdot \text{s}$)
$$
t_0 = \frac{3\mu}{\gamma h_0^3 q_0^4} \approx 300 \, \text{s}
$$
and it can be seen that indeed film rupture occurs within time 
scales of the order of $t_0$ ($\approx 2 t_0$) \cite{becker2003complex}.
We have reported these comments on comparison with actual experimental 
values in the new version of the paper as follows:\\
\\
\textcolor{red}{To give a flavour of how to compare with actual experimental systems, let us notice
first that the expression of $t_0$ depends on the disjoining pressure (through its dependence 
 $q_0 = \sqrt{\frac{1}{2\gamma} \Pi^{\prime}(h_0)}$) and contains, therefore, energetic details, namely 
 the Hamaker constant $A_H$. Consider, for instance, the problem studied in \cite{becker2003complex}: the dewetting of a polystyrene film of thickness $h_0 \approx 4 \, \text{nm}$ deposited on a silicon 
dioxide substrate, where $A_H  = 2.2 \times 10^{-20} \, \text{J}$ and 
$\gamma = 0.03 \, \text{N}/\text{m}$. Neglecting the short range part of the interface potential
(such that $\Pi(h)= - \frac{A_H}{6\pi h^3}$)~\cite{Mecke_2005,becker2003complex}, one gets 
$q_0^2 \approx \frac{A_H}{4\pi \gamma h_0^4} \approx 2.5 \times 10^{-4} \, \text{nm}^{-2}$.
The characteristic time of the growth of the instability for these experiments, then, reads \cite{PhysRevLett.99.114503} (the dynamic viscosity of polystyrene is 
$\mu = 1.2 \times 10^4 \, \text{Pa} \cdot \text{s}$) $t_0 = \frac{3\mu}{\gamma h_0^3 q_0^4} \approx 300 \, \text{s}$ and it can be seen that indeed film rupture occurs within time 
scales of the order of $t_0$ ($\approx 2 t_0$) \cite{becker2003complex}, as in our 
simulation for the largest pattern wavelength, $\lambda = L$ (see also our previous work
with homogeneous substrates~\cite{PhysRevE.104.034801}).} 
\\

Concerning the values of viscosity and surface tension in lattice Boltzmann units, 
we acknowledge that giving them in the middle of the discussion on characteristic length and time might  
have caused confusion. Nevertheless we deem important to provide them, for the sake of readers interested 
in numerical details of the method as well as for the results reproducibility, therefore we have moved them 
to a later paragraph, where we added: \\
\\
\textcolor{red}{The numerical values in lattice Boltzmann units of the 
mean film thickness, dynamics viscosity and surface tension are, respectively, $h_0=1$, $\mu=1/6$
and $\gamma=0.01$.}
}


\item[ \textbf{\underline{Comment 6.}}]
{ 
What is the disjoining pressure due to? 
Van der Waals forces? 
How thick is the precursor film, and why should the disjoining pressure be zero there?
Is there any comparison with experiments possible/feasible?
}

\item[ \textbf{{Answer}}]
{
As we have already discussed in the first answer (and as one can find 
in several reviews and undergraduate textbooks, but see, e.g., Craster 
and Matar \cite{craster2009dynamics}), the disjoining pressure encode 
any kind of interactions among the fluid molecules and the substrate, so 
also those of Van der Waals type. The precursor film is an equilibrium value for the thickness that emerges when the interface potential has also a short 
range repulsive term. The disjoining pressure is zero there by definition of equilibrium (or, equivalently, because that is a minimum of the interface potential). Concerning comparisons of a thin-film equation approach with 
experiments, there are tons of examples in the literature, such as 
\cite{becker2003complex,fetzer2005new,PhysRevLett.99.114503} to name but a few (see also \cite{craster2009dynamics,RevModPhys.81.739} for reviews). All this literature is cited in the paper.
}

\end{itemize}


\bibliographystyle{abbrv}
\bibliography{Ref}

\end{document}
